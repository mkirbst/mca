%Wir verwenden eine DIN-A4-Seite und die Schriftgröße 12.
\documentclass[a4paper,12pt]{scrartcl} 


%Diese drei Pakete benötigen wir für die Umlaute, Deutsche Silbentrennung etc.
%Apple-Nutzer sollten anstelle von \usepackage[latin1]{inputenc} das Paket \usepackage[applemac]{inputenc} verwenden
%% \usepackage[latin1]{inputenc}
%%apt-get install texlive-lang-german damit ngerman keine Probleme mehr macht !!
%\usepackage[utf8]{inputenc} 
%\usepackage[T1]{fontenc}
%\usepackage[ngerman]{babel}

%Das Paket erzeugt ein anklickbares Verzeichnis in der PDF-Datei.
%\usepackage{hyperref}

%Das Paket wird für die anderthalb-zeiligen Zeilenabstand benötigt
\usepackage{setspace}

%%HTWM-Vorlage - benoetigt apt-get install texlive-fonts-extra
\setcounter{tocdepth}{3}				%Schatelungstiefe Inhaltsverz.
\usepackage[utf8]{inputenc}			%deutsche Umlaute
\usepackage{german, ngerman}
\usepackage[ngerman]{babel}			%Rechtschreibprüfung
\usepackage{color,listings} 			%Quellcode Highlighting, bindet das
%Paket Listings ein
\usepackage{listings}
\usepackage{color}
\usepackage{textcomp}
\usepackage[T1]{fontenc}				%srccode
\usepackage[scaled]{beramono}		%srccode
\usepackage{longtable}				%mehrseitige tabellen
\usepackage[tableposition=b]{caption}
\usepackage[pdftex, pdftoolbar=false, hyperfootnotes=false, bookmarks,
bookmarksopen, bookmarksnumbered, bookmarksopenlevel=2, pdfpagelabels=true,
pdfstartpage=3, pdfstartview=FitH,]{hyperref} %Verlinkungen
\usepackage{array}					%farbige Tabellen
\usepackage[table]{xcolor} 			%farbige Tabellen
\usepackage{graphicx}				% \includegraphics bnoetigt dies

\usepackage{fancyhdr, graphicx}		%% Logo auf Titelseite
\renewcommand{\headrulewidth}{0pt}
\fancyhead[L]{}
\fancyhead[R]{
  \includegraphics[width=52mm]{./images/htwk.png}
}
\usepackage{draftwatermark}			% wasserzeichen
	%Quelle: http://choorucode.com/2010/05/05/latex-adding-draft-watermark/?like=1&source=post_flair&_wpnonce=1c9f85538d
\SetWatermarkText{VORABVERSION}		% wasserzeichen-text
\SetWatermarkLightness{0.9}			% wasserzeichen-kontrast
\SetWatermarkScale{2.5}				% wasserzeichen-zeichengroe\ss{}e



\definecolor{Navy}{rgb}{0,0,0.5}
\definecolor{Gray}{gray}{0.5}
\definecolor{dunkelgrau}{rgb}{0.8,0.8,0.8}
\definecolor{hellgrau}{rgb}{0.95,0.95,0.95}
\definecolor{hellgrau2}{rgb}{0.93,0.93,0.93}

\hypersetup{
	colorlinks=true, 			% false: boxed links; true: colored links
	linkcolor=Navy,          		% color of internal links
	citecolor=Gray,        			% color of links to bibliography
	filecolor=magenta,      		% color of file links
	urlcolor=blue,           			% color of external links
	linkbordercolor={1 1 1}, 		% set to white
	citebordercolor={1 1 1} 		% set to white
}


%Einrückung eines neuen Absatzes
\setlength{\parindent}{0em}

%Definition der Ränder
\usepackage[paper=a4paper,left=30mm,right=30mm,top=30mm,bottom=30mm]{geometry}

%Abstand der Fussnoten
\deffootnote{1em}{1em}{\textsuperscript{\thefootnotemark\ }}

%Regeln, bis zu welcher Tiefe (section,subsection,subsubsection) Überschriften angezeigt werden sollen (Anzeige der Überschriften im Verzeichnis / Anzeige der Nummerierung)
%\setcounter{tocdepth}{3}
%\setcounter{secnumdepth}{3}

\fancypagestyle{htwkheader}
{
   \fancyhf{}	% clear all header and footer fields
  \fancyhead[RO]{
	\makebox[\textwidth]{	%% schiebe Logo nach aussen auf den Rand
		\rule{0.9				%% nach aussen schieben hoeherer Wert -> Logo weiter nach aussen
		  \textwidth}{0cm} %% nicht nach unten schieben = 0cm
			\includegraphics*[width=52mm]{./images/htwk.png}	%%Logo HTWK
	  }
  }
}


\begin{document}
%Beginn der Titelseite

\begin{titlepage}
\thispagestyle{htwkheader}		
%\addtolength{\voffset}{-2cm}		
%\addtolength{\topmargin}{-2cm}
%\addtolength{\bottommargin}{2cm}
%%%%%%%%%%%%%%%%%%%%%%%%%%%%%%%%%%%%%%%%%%%%%%%%%%%%%%%%%%
%%  Oberer Teil: Links Textblock HTWK, Rechts Logo HTWK %%
%%\begin{figure}[htbp]
%%\begin{minipage}[t]{6cm}	%% linker Teil
%%\vspace{0cm}
HTWK Leipzig\\
Fachbereich IMN \\
Wintersemester 2012/2013
%%\end{minipage}
%%\hfill						%% Zwischenraum auffuellen
%%\begin{minipage}[t]{6cm}	%% rechter Teil
%%\vspace{0pt}
%%\makebox[\textwidth]{	%% schiebe Logo nach aussen auf den Rand
%%  \rule{1				%% nach aussen schieben hoeherer Wert -> Logo weiter nach aussen
%%    \textwidth}{0cm} %% nicht nach unten schieben = 0cm
%%      \includegraphics*[width=52mm]{./images/htwk.png}	%%Logo HTWK
%%}

%%  \end{flushright}
%%\caption{Bild1}			%% keine Bildbeschriftung fuer Logo
%%\label{fig:Bild1}			%% nicht ins Abbildungsverzeichnis aufnehmen
%%\end{minipage}
%%\end{figure}

%\vspace{6cm}

%\addtolength{\voffset}{0cm}

\begin{center}
\begin{Large}
\vfill {\textsf{\textbf{
Beleuchtungssteuerung mit dem Mikrocontroller LPC1768
\\--VORABVERSION--\\
}}}
\end{Large}
Beleg im Mikrocontrolleranwendungen
\end{center}

\begin{small}
\vfill
Marcel Kirbst \\
Sieglitz 39 \\
06618 Molau \\
marcel.kirbst@stud.htwk-leipzig.de\\
\\
Sebastian Krause\\
Dante-Stra\ss{}e 16 \\
04159 Leipzig \\
sebastian.krause@stud.htwk-leipzig.de\\
\\
\today
\end{small}

\end{titlepage}
\addtolength{\voffset}{0cm}
%%%%%%%%%%%%%%%%%%%%%%%%%
%%%Ende der Titelseite%%%
%%%%%%%%%%%%%%%%%%%%%%%%%


%Inhaltsverzeichnis (aktualisiert sich erst nach dem zweiten Setzen)
\tableofcontents
\thispagestyle{empty}

%Beginn einer neuen Seite
\clearpage

%Anderthalbzeiliger Zeilenabstand ab hier
\onehalfspacing

%%\pagestyle{plain}
\pagestyle{headings}	%% lebende Kopfzeile

%%Abbildungsverzeichnis hier erstellen
\clearpage
\listoffigures

\clearpage
\section{Einleitung}
Dieser Beleg befasst sich mit der Helligkeitssteuerung von Leuchtmodulen durch den Mikrocontroller LPC1768 in Verbindung mit dem PWM-Treiber PCA9685. Bei den
Leuchtmodluen handelt es sich um Baugruppen die mit jeweils sechs 1Watt-LEDs best\"uckt sind und \"uber eine integrierte Transistor-Endstufe versorgt werden.  
Weiterhin soll ermittelt werden wie die Aussteuerung der einzelnen PWM-Stufen mit der real messbaren Beleuchtungsst\"arke korreliert. 
%% BILD:
%\begin{figure}[htb]
%\begin{center}
% \includegraphics[width=0.65\hsize]{./images/default.pdf}
%\end{center}
%\caption[Beispielhafte Standardkonfiguration eines Internetanschlu\ss{}, Quelle: Autor, verwendete Symbole unterliegen der
%GPL]{\label{stdinet}Beispielhafte Standardkonfiguration eines Internetanschlu\ss{}.}
%\end{figure}

\clearpage
\section{Grundlagen}
\subsection{I2C}

\clearpage
\section{Eingesetzte Hardware}
\subsection{LPC1768}
\subsection{PCA9685}

\clearpage
\section{Implementierung}
\subsection{Quellcode}

\clearpage
\section{Auswertung}
\subsection{Messergebisse}


%% Tabelle

%\vspace*{1cm}
%\begin{longtable}{p{34mm}>{\columncolor[gray]{0.97}}p{33mm}p{33mm}>{\columncolor[gray]{0.97}}p{33mm}}
%\rowcolor[gray]{.9}Funktion & \textbf{IPCop} & \textbf{IPFire} & \textbf{pfSense}\\
%Lizenz & GPL\cite{GPLLicense} & GPL\cite{GPLLicense} & BSD\cite{FreeBSDLicense}\\
%\rowcolor[gray]{.95}Betriebssystem & Linux & Linux & FreeBSD   \\
%Hardware"-architektur & i386, Cobald, Sparc, PowerPC & i386, AMD64 & i386, AMD64\\
%\caption{Merkmale ausgew\"ahlter Routerdistributionen im Vergleich}
%\label{Merkmale der Routerdistributionen im Vergleich}
%\end{longtable}

\clearpage
\section{Schluss}
Dies ist der Schlussteil. Abschlie\ss{}ende Empfehlung

\clearpage
\section{Glossar}
\begin{description}
 \item[I2C] Prtokoll zur Kommunikation in Ger\"aten
\end{description}

\clearpage
\section{Literatur- und Quellenverzeichnis}

\renewcommand\refname{Literaturverzeichnis}
\begin{thebibliography}{999}

\bibitem{buch_foobar}Michael W. Lucas:  {\sl Absolute BSD (2nd Edition). The Ultimate Guide to FreeBSD.} No Starch Press, 2008,
\\ISBN: 978-1-59327-151-0

\end{thebibliography}

\renewcommand\refname{Quellenverzeichnis}
\begin{thebibliography}{999}

%%\cite{citeverweis}
\bibitem{citeverweis}
\url{http://www.citeverweis.foo.bar}
\\Abrufbar am 25.02.2013.

\end{thebibliography}

\clearpage
\section{Verzichtserkl\"arung}
\thispagestyle{plain}

Hiermit erkläre ich, dass ich die vorliegende Arbeit selbstständig und nur unter Verwendung der angegebenen Literatur und Hilfsmittel angefertigt habe.
Stellen, die wörtlich oder sinngemäß aus Quellen entnommen wurden, sind als solche
kenntlich gemacht.\\

Diese Arbeit wurde in gleicher oder ähnlicher Form noch keiner anderen Prüfungsbehörde vorgelegt.\\\\

Leipzig, \today
\end{document}
%% EOF